% Chapter Template
% Copy this file for each chapter: chapter01.tex, chapter02.tex, etc.

\chapter{Chapter Title}
\label{ch:chapterlabel}

% Brief chapter introduction
This chapter presents [what this chapter is about]. Section~\ref{sec:section1} discusses [topic 1], Section~\ref{sec:section2} covers [topic 2], and Section~\ref{sec:conclusion} concludes.

% ============================================================================
\section{First Section}
\label{sec:section1}

This section introduces [topic].

% Subsection example
\subsection{Subsection Example}
\label{subsec:example}

Content goes here with citations \cite{smith2023example}.

% Equation example
The key equation is:
\begin{equation}
\label{eq:example}
    f(x) = \frac{1}{{\sigma \sqrt {2\pi } }}e^{-\frac{(x - \mu)^2}{2\sigma^2}}
\end{equation}
where $\mu$ is the mean and $\sigma$ is the standard deviation.

As shown in Equation~\ref{eq:example}, [explain].

% ============================================================================
\section{Second Section}
\label{sec:section2}

This section covers [topic].

% Figure example
\begin{figure}[htbp]
    \centering
    \includegraphics[width=0.8\textwidth]{example_figure.pdf}
    \caption{Description of the figure. This should be self-contained.}
    \label{fig:example}
\end{figure}

As shown in Figure~\ref{fig:example}, [describe what the figure shows].

% Table example
\begin{table}[htbp]
    \centering
    \caption{Results comparison. Tables should have captions above.}
    \label{tab:results}
    \begin{tabular}{lccc}
        \toprule
        Method & Metric 1 & Metric 2 & Metric 3 \\
        \midrule
        Baseline 1 & 85.3 & 72.1 & 68.9 \\
        Baseline 2 & 87.2 & 74.5 & 70.3 \\
        \textbf{Our Method} & \textbf{92.1} & \textbf{81.3} & \textbf{76.8} \\
        \bottomrule
    \end{tabular}
\end{table}

Table~\ref{tab:results} shows that [describe results].

% Algorithm example
\begin{algorithm}
    \caption{Algorithm Name}
    \label{alg:example}
    \begin{algorithmic}[1]
        \REQUIRE Input: $x$
        \ENSURE Output: $y$
        \STATE Initialize $y \leftarrow 0$
        \FOR{$i = 1$ to $n$}
            \STATE $y \leftarrow y + f(x_i)$
        \ENDFOR
        \RETURN $y$
    \end{algorithmic}
\end{algorithm}

Algorithm~\ref{alg:example} describes [what it does].

% ============================================================================
\section{Conclusion}
\label{sec:conclusion}

This chapter presented [summary of contributions]. We showed [key findings]. The next chapter will [preview next chapter].
